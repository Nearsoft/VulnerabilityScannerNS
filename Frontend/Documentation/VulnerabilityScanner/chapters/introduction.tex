\chapter{Introduction}
Users trust web services with sensitive information, like credit card information,  birthdays, and addresses. This makes Web Applications a high-value target for bad intentioned folks also known as Black Hat Hackers.

It is the duty of the service provider to assure that the user information will be secure, this responsibility ends up in the hands of the developers.

This tool aims to help developers safeguard their Web Applications and Services providing a vulnerability scanner that can be used to find well known security holes.

It is designed to find various vulnerabilities using "black-box" method, that means it won't study the source code of web applications but will work using fuzz testing, scanning the pages of the deployed web application, extracting links and forms and attacking the scripts.

\section{What?}

The objective of this project is to deliver a web platform where users can check if a web page has ten of the Web Application Security Risks, by providing and scanning the url of the web page. The platform will contain a dashboard for the user, where it will show the scans results, represented by graphs and text, and also the user will have the option to export the report in .PDF, .XML and .TXT formats, or keep consulting it at the web platform by logging with his user account \cite{owasp}.

\section{Why?}
The importance of delivering a web platform that can scan the security risks of other web platforms, is to have a powerful tool that can be used for testing new websites and verify if a website is exposed to the most common security risks.


\section{How?}
The software will have several components, it will connect to Python REST API to perform certain tasks to test web vulnerabilities that could take a lot of time to finish.The application itself will be developed using Ruby On Rails, Python and PostgreSQL to manage data.

\section{Limitations}
Due to the time and the way some vulnerabilities checks have to be done (access to the website’s source code and/or the server), we consider that it is not feasible to implement some vulnerabilities or to extend the functionality of the already implemented ones.

\section{Vulnerabilities}
The vulnerabilities tested by this application are the following ones: 

\begin{itemize}
    \item XSS - Cross Site Scripting
    \item DOM based XSS
    \item Blind SQL Injection
    \item SQL Injection
    \item HTTP Header SQL Injection
    \item PHP Injection
    \item OS Command Injection
    \item HTML Injection
    \item XML Injection 
    \item LDAP Injection
\end{itemize}



